\documentclass[a4paper,12pt]{article}
\usepackage[utf8]{inputenc}
\usepackage[ngerman]{babel}
\usepackage{tikz}
\usepackage{fancyhdr}
\usepackage{listings}
\usepackage{xcolor}
\usepackage{amsmath}
\usepackage{tikz-timing}
\usepackage{circuitikz}
\usepackage{float}
\usepackage{pgfplots}
\pgfplotsset{compat=1.18}
\usepackage{filecontents}
\usepackage{hyperref}
\usepackage{dirtree}
\usepackage{inconsolata}



\setlength{\headheight}{45.25728pt}
\addtolength{\topmargin}{-33.25728pt}
\addtolength{\topmargin}{-25.03186pt}
\pagestyle{fancy}
\fancyhf{} % Leert Standard Kopf- und Fußzeilen



% Fußlinie aktivieren (Stärke anpassen)
\renewcommand{\footrulewidth}{0.4pt}

% Kopfzeile
\fancyhead[L]{\includegraphics[width=2.5cm]{logo.png}}    % Logo links
\fancyhead[C]{14.10.2025}                               % Datum in der Mitte
\fancyhead[R]{Felix Zauner}                             % Platzhalter für Namen

% Fußzeile
\fancyfoot[L]{5AHETS}                                  % Klasse links
\fancyfoot[C]{Labor}                                   % Mitte Labor
\fancyfoot[R]{Seite \thepage}                          % Seitenzahl rechts

% Define custom colors
\definecolor{vscodeblue}{RGB}{86, 156, 214}          % VSCode blue for keywords
\definecolor{vscodepurple}{RGB}{197, 134, 192}       % VSCode purple for strings
\definecolor{vscodegreen}{RGB}{78, 201, 176}         % VSCode green for comments
\definecolor{vscodeorange}{RGB}{220, 165, 95}        % VSCode orange for numbers
\definecolor{vscodegray}{RGB}{155, 155, 155}         % VSCode gray for line numbers

\begin{document}

\lstset{
    % Font settings (Inconsolata is very similar to VSCode's default)
    basicstyle=\ttfamily\footnotesize,
    % Colors and background
    backgroundcolor=\color{white},
    rulecolor=\color{gray!50},
    % Line numbers
    numbers=left,
    numberstyle=\tiny\color{vscodegray},
    numbersep=8pt,
    % Frame
    frame=single,
    framerule=0.5pt,
    framesep=5pt,
    % Syntax highlighting colors
    commentstyle=\color{vscodegreen},
    keywordstyle=\color{vscodeblue},
    stringstyle=\color{vscodepurple},
    % Layout
    tabsize=4,
    showspaces=false,
    showstringspaces=false,
    showtabs=false,
    breaklines=true,
    breakatwhitespace=true,
    captionpos=b,
    xleftmargin=10pt,
    xrightmargin=5pt
}

% Titelseite
\begin{titlepage}
    \centering
    \vspace*{2cm}
    % Logo auf Titelseite
    \includegraphics[width=0.5\textwidth]{logo.png}

    \vspace{1.5cm}
    % Titel und Untertitel
    \begin{minipage}{0.9\textwidth}
        \centering
        \Huge\textbf{Automatische Klassifikation von Thorax-Röntgenbildern zur Erkennung von Pneumonie und Tuberkulose}\\
        \vspace{0.5cm}
        
    \end{minipage}
    \Large Projektbericht
    \vspace{1.5cm}

    % Autoren
    \textbf{Gruppe:}\\
    Felix Zauner, Timofey Luzin\\

    \vspace{0.5cm}

    \textbf{Klasse \& Schuljahr:} \\ 5AHETS 2025/26


    \vspace{0.5cm}

    \textbf{Abgabe:} \\ 15.12.2025


\end{titlepage}



\tableofcontents

\newpage
\section{Einleitung}
In diesem Projektbericht wird die Entwicklung eines Modells zur automatischen Klassifikation von Thorax-Röntgenbildern zur Erkennung von Pneumonie und Tuberkulose beschrieben. 
\\
Die KI-basierte Bildanalyse hat in den letzten Jahren erhebliche Fortschritte gemacht und bietet vielversprechende Möglichkeiten zur Unterstützung medizinischer Diagnosen.
Manche Experten sehen in der automatischen Bildanalyse sogar das Potenzial, die Genauigkeit und Effizienz von Diagnosen zu verbessern, insbesondere in ressourcenarmen Umgebungen.
\\
Ziel dieses Projekts ist es, ein Modell zu entwickeln, das in der Lage ist, Thorax-Röntgenbilder zu analysieren und zwischen gesunden Patienten, Patienten mit Pneumonie und Patienten mit Tuberkulose zu unterscheiden.
\\
Der Bericht gliedert sich in mehrere Abschnitte, die den gesamten Entwicklungsprozess abdecken, von der Datensammlung und -aufbereitung über die Modellauswahl und das Training bis hin zur Evaluation des Modells.
Abschließend werden die Herausforderungen und Probleme, die während des Projekts aufgetreten sind, sowie eine Reflexion über die Ergebnisse und mögliche zukünftige Verbesserungen diskutiert.


\newpage

\section{Datensatzbeschreibung}
Der verwendete Datensatz stammt von Kaggle und enthält Thorax-Röntgenbilder, die in drei Kategorien unterteilt sind: gesund, Pneumonie und Tuberkulose.
Die Bilder sind in verschiedenen Auflösungen und Formaten vorhanden, was eine einheitliche Datenaufbereitung erforderlich macht. \\ \\
Hier ist eine kurze Übersicht über die Struktur des Datensatzes:

\begin{figure}[H]
    \dirtree{%
    .1 chest-xray-dataset.
    .2 versions.
    .3 1.
    .4 test.
    .5 normal.
    .6 [925 images].
    .5 pneumonia.
    .6 [580 images].
    .5 tuberculosis.
    .6 [1064 images].
    .4 train.
    .5 normal.
    .6 [7263 images].
    .5 pneumonia.
    .6 [4674 images].
    .5 tuberculosis.
    .6 [8513 images].
    .4 val.
    .5 normal.
    .6 [900 images].
    .5 pneumonia.
    .6 [570 images].
    .5 tuberculosis.
    .6 [1064 images].
    .4 data.yaml.
    }
    \caption{Struktur des Datensatzes}
    \label{fig:datensatzstruktur}
\end{figure}



\newpage

\section{Datenaufbereitung}
Aufgrund der unterschiedlichen Auflösungen der Bilder im Datensatz war eine sorgfältige Datenaufbereitung erforderlich, um eine konsistente Eingabe für das Modell zu gewährleisten. \\
Die Bilder wurden sowohl auf eine einheitliche Größe von 128x128 Pixel skaliert, als auch in schwarz-weiß umgewandelt, um die Verarbeitung zu erleichtern und die Leistung des Modells zu optimieren. \\

\begin{lstlisting}[language=Python, caption=Datenaufbereitungscode]
import cv2
# Transform and load images
def load_images(folder, label, img_size=(128, 128)):
    X, y = [], []

    for file in os.listdir(folder):
        path = os.path.join(folder, file)
        img = cv2.imread(path, cv2.IMREAD_GRAYSCALE)
        img = cv2.resize(img, img_size)
        X.append(img)
        y.append(label)

    return X, y
\end{lstlisting}

\newpage

\section{Modellauswahl und -training}
\subsection{Modellauswahl}
Es wurden alle möglichen Modelle ausprobiert

\subsection{Modelltraining}
\subsection{Code-Dokumentation}
\subsection{Probleme und Herausforderungen}

\newpage

\section{Evaluation des Modells}

\newpage

\section{Probleme und Herausforderungen im gesamten Projekt}
\subsection{Technische Herausforderungen}
\subsection{Modellierungsprobleme}

\newpage

\section{Schlussfolgerung und Reflexion}

\newpage

\section{Quellen und Literaturverzeichnis}

\newpage

\section{Anhang}

\end{document}