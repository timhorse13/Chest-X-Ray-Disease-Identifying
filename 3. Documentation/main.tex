\documentclass[a4paper,12pt]{article}
\usepackage[utf8]{inputenc}
\usepackage[ngerman]{babel}
\usepackage{tikz}
\usepackage{fancyhdr}
\usepackage{listings}
\usepackage{xcolor}
\usepackage{amsmath}
\usepackage{tikz-timing}
\usepackage{circuitikz}
\usepackage{float}
\usepackage{pgfplots}
\pgfplotsset{compat=1.18}
\usepackage{filecontents}




\setlength{\headheight}{45.25728pt}
\addtolength{\topmargin}{-33.25728pt}
\addtolength{\topmargin}{-25.03186pt}
\pagestyle{fancy}
\fancyhf{} % Leert Standard Kopf- und Fußzeilen



% Fußlinie aktivieren (Stärke anpassen)
\renewcommand{\footrulewidth}{0.4pt}

% Kopfzeile
\fancyhead[L]{\includegraphics[width=2.5cm]{logo.png}}    % Logo links
\fancyhead[C]{14.10.2025}                               % Datum in der Mitte
\fancyhead[R]{Felix Zauner}                             % Platzhalter für Namen

% Fußzeile
\fancyfoot[L]{5AHETS}                                  % Klasse links
\fancyfoot[C]{Labor}                                   % Mitte Labor
\fancyfoot[R]{Seite \thepage}                          % Seitenzahl rechts

\begin{document}

% Titelseite
\begin{titlepage}
    \centering
    \vspace*{2cm}
    % Logo auf Titelseite
    \includegraphics[width=0.5\textwidth]{logo.png}

    \vspace{1.5cm}
    % Titel und Untertitel
    \begin{minipage}{0.9\textwidth}
        \centering
        \Huge\textbf{Automatische Klassifikation von Thorax-Röntgenbildern zur Erkennung von Pneumonie und Tuberkulose}\\
        \vspace{0.5cm}
        
    \end{minipage}
    \Large Projektbericht
    \vspace{1.5cm}

    % Autoren
    \textbf{Gruppe:}\\
    Felix Zauner, Timofey Luzin\\

    \vspace{0.5cm}

    \textbf{Klasse \& Schuljahr:} \\ 5AHETS 2025/26


    \vspace{0.5cm}

    \textbf{Abgabe:} \\ 15.12.2025


\end{titlepage}



\tableofcontents

\newpage
\section{Einleitung}
In diesem Projektbericht wird die Entwicklung eines Modells zur automatischen Klassifikation von Thorax-Röntgenbildern zur Erkennung von Pneumonie und Tuberkulose beschrieben. 
\\
Die KI-basierte Bildanalyse hat in den letzten Jahren erhebliche Fortschritte gemacht und bietet vielversprechende Möglichkeiten zur Unterstützung medizinischer Diagnosen.
Manche Experten sehen in der automatischen Bildanalyse sogar das Potenzial, die Genauigkeit und Effizienz von Diagnosen zu verbessern, insbesondere in ressourcenarmen Umgebungen.
\\
Ziel dieses Projekts ist es, ein Modell zu entwickeln, das in der Lage ist, Thorax-Röntgenbilder zu analysieren und zwischen gesunden Patienten, Patienten mit Pneumonie und Patienten mit Tuberkulose zu unterscheiden.
\\
Der Bericht gliedert sich in mehrere Abschnitte, die den gesamten Entwicklungsprozess abdecken, von der Datensammlung und -aufbereitung über die Modellauswahl und das Training bis hin zur Evaluation des Modells.
Abschließend werden die Herausforderungen und Probleme, die während des Projekts aufgetreten sind, sowie eine Reflexion über die Ergebnisse und mögliche zukünftige Verbesserungen diskutiert.


\newpage

\section{Datensatzbeschreibung}

\newpage

\section{Datenaufbereitung}

\newpage

\section{Modellauswahl und -training}
\subsection{Modellauswahl}
\subsection{Modelltraining}
\subsection{Code-Dokumentation}
\subsection{Probleme und Herausforderungen}

\newpage

\section{Evaluation des Modells}

\newpage

\section{Probleme und Herausforderungen im gesamten Projekt}
\subsection{Technische Herausforderungen}
\subsection{Modellierungsprobleme}

\newpage

\section{Schlussfolgerung und Reflexion}

\newpage

\section{Quellen und Literaturverzeichnis}

\newpage

\section{Anhang}

\end{document}